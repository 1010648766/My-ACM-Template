\textbf{上下界网络流建图方法}
\subsubsection*{记号说明}
\begin{itemize}
\item $f(u,v)$表示$u \rightarrow v$的实际流量
\item $b(u,v)$表示$u \rightarrow v$的流量下界
\item $c(u,v)$表示$u \rightarrow v$的流量上界
\end{itemize}

\subsubsection*{无源汇可行流}
\paragraph{建图}
\begin{itemize}
\item 新建附加源点$S$和$T$
\item 原图中的边$u \rightarrow v$,限制为$[b,c]$,建边$u \rightarrow v$,容量为$c-b$
\item 记$d(i) = \sum b(u,i) - \sum b(i,v) $
\item 若$d(i)>0$,建边$S \rightarrow i$,流量为$d(i)$
\item 若$d(i)<0$,建边$i \rightarrow T$,流量为$-d(i)$
\end{itemize}

\paragraph{求解}
\begin{itemize}
\item 跑$S \rightarrow T$的最大流,如果满流,则原图存在可行流。
\item 此时,原图中每一条边的流量为新图中对应边的流量加上这条边的下界。
\end{itemize}

\subsubsection*{有源汇可行流}
\paragraph{建图}
\begin{itemize}
\item 在原图中建边$t \rightarrow s$,流量限制为$[0,+\infty)$,这样就改造成了无源汇的网络流图。
\item 之后就可以像求解无源汇可行流一样建图了。
\end{itemize}

\paragraph{求解} 同无源汇可行流

\subsubsection*{有源汇最大流}
\paragraph{建图} 同有源汇可行流
\paragraph{求解}
\begin{itemize}
\item 先跑一遍$S \rightarrow T$的最大流,求出可行流
\item 记此时$\sum f(s,i) = sum_1$
\item 将$t \rightarrow s$这条边拆掉,在新图上跑$s \rightarrow t$的最大流
\item 记此时$\sum f(s,i) = sum_2$
\item 最终答案即为$sum_1 + sum_2$
\end{itemize}

\subsubsection*{有源汇最小流}
\paragraph{建图} 同无源汇可行流
\paragraph{求解}
\begin{itemize}
\item 求$S \rightarrow T$最大流
\item 建边$t \rightarrow s$,容量为$+\infty$
\item 再跑一遍$S \rightarrow T$的最大流,答案即为$f(t,s)$
\end{itemize}

\textbf{有源汇的最大流和最小流也可以通过二分答案求得,\\即二分$t \rightarrow s$的下界(最大流)和上界(最小流)复杂度多了个$O(\log n)$这里不再赘述。}

\paragraph{蓝书上的做法}
\begin{itemize}
\item 先用无源汇可行流建图的方法求出可行流,然后用传统$s-t$增广路算法即可得到最大流。把$t$看成源点,$s$看成汇点后求出的$t-s$最大流就是最小流。
\item 注意:原先每条弧$u \rightarrow v$的反向弧容量为$0$,而在有容量下界的情形中,反向弧的容量应该等于流量下界。
\end{itemize}

\subsubsection*{有源汇费用流}
\paragraph{建图}
\begin{itemize}
\item 新建附加源点$S$和$T$
\item 原图中的边$u \rightarrow v$,限制为$[b,c]$,费用为$cost$,建边$u \rightarrow v$,容量为$c-b$,费用为$cost$
\item 记$d(i) = \sum b(u,i) - \sum b(i,v) $
\item 若$d(i)>0$,建边$S \rightarrow i$,流量为$d(i)$,\textbf{费用为$0$}
\item 若$d(i)<0$,建边$i \rightarrow T$,流量为$-d(i)$,\textbf{费用为$0$}
\item 建边$t \rightarrow s$,流量为$+\infty$,费用为$0$。
\end{itemize}

\paragraph{求解}
\begin{itemize}
\item 跑$S \rightarrow T$的最小费用最大流
\item 答案为求出的费用加上原图中边的下界乘以边的费用
\end{itemize}
