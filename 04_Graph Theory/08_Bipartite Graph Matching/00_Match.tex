\begin{enumerate}
\item 一个二分图中的最大匹配数等于这个图中的最小点覆盖数

\item 最小路径覆盖=$|G|$-最大匹配数

在一个 $N \times N$ 的有向图中,路径覆盖就是在图中找一些路经,使之覆盖了图中的所有顶点,且任何一个顶点有且只有一条路径与之关联;

(如果把这些路径中的每条路径从它的起始点走到它的终点,那么恰好可以经过图中的每个顶点一次且仅一次);如果不考虑图中存在回路,那么每每条路径就是一个弱连通子集.

由上面可以得出:

\begin{enumerate}
\item 一个单独的顶点是一条路径;

\item 如果存在一路径 $p_1,p_2,......p_k$,其中 $p_1$ 为起点,$p_k$ 为终点,那么在覆盖图中,顶点 $p_1,p2,......p_k$ 不再与其它的顶点之间存在有向边.
\end{enumerate}

最小路径覆盖就是找出最小的路径条数,使之成为 $G$ 的一个路径覆盖.

路径覆盖与二分图匹配的关系:最小路径覆盖=$|G|$-最大匹配数;

\item 二分图最大独立集=顶点数-二分图最大匹配

独立集:图中任意两个顶点都不相连的顶点集合。
\end{enumerate}
